
\documentclass[a4paper]{article}
\usepackage[margin=1cm]{geometry}
\usepackage{amsmath}
\usepackage{amssymb}
\usepackage{mathtools}
\usepackage{booktabs}


\title{Formal analysis of the $E/A_{-}$ function}
\author{}
\date{}

\begin{document}

\maketitle

\section{Function Definition}

We will simply denote $E/A_{-}$ by $E$ throughout this document. The function is defined as:
\begin{align}
E(r,t,I,S) &= t_1(r,t,I,S) + t_2(r,t,I,S)\\
\end{align}
where:
\begin{align}
t_1(r,t,I,S) &= -2\sqrt{rt(2I-r)(2S-t)}\\
t_2(r,t,I,S) &= \frac{I-1}{2S}t + \frac{S-1}{2I}r + \frac{r}{2IS}t
\label{terms}
\end{align}

For fixed parameters $I$ and $S$, we can express the function as:
\begin{align}
E(r,t) &= -2\sqrt{rt(2I-r)(2S-t)} + \frac{I-1}{2S}t + \frac{S-1}{2I}r + \frac{r}{2IS}t
\end{align}

\section{Critical Points}
The critical points of the $E$ function are found by setting the first-order partial derivatives equal to zero:
\begin{align}
\frac{\partial E}{\partial r} &= \frac{-t(2I-r)(2S-t)}{2r\sqrt{rt(2I-r)(2S-t)}} + \frac{S-1}{2I} + \frac{t}{2IS} = 0\ ,\nonumber\\
\frac{\partial E}{\partial t} &= \frac{-r(2I-r)(2S-t)}{2t\sqrt{rt(2I-r)(2S-t)}} + \frac{I-1}{2S} + \frac{r}{2IS} = 0
\label{system-eq}
\end{align}
These equations determine the critical points $(r_c, t_c)$ where the gradient of the function is zero.

\section{Hessian Matrix}

The Hessian matrix of the $E$ function with respect to variables $r$ and $t$ (treating $I$ and $S$ as fixed parameters) is:

\begin{align}
H_{E}(r,t) = 
\begin{pmatrix}
\frac{\partial^2 E}{\partial r^2} & \frac{\partial^2 E}{\partial r \partial t} \\
\frac{\partial^2 E}{\partial t \partial r} & \frac{\partial^2 E}{\partial t^2}
\end{pmatrix}
= 
\begin{pmatrix}
\frac{t(2S-t)(3r-2I)}{4r(r-2I)^2\sqrt{rt(2I-r)(2S-t)}} & \frac{1}{2IS} - \frac{(2I-r)(2S-t) + rt}{4rt\sqrt{rt(2I-r)(2S-t)}} \\
\frac{1}{2IS} - \frac{(2I-r)(2S-t) + rt}{4rt\sqrt{rt(2I-r)(2S-t)}} & \frac{r(2I-r)(3t-2S)}{4t(t-2S)^2\sqrt{rt(2I-r)(2S-t)}}
\end{pmatrix}
\end{align}

\section{Function Domain}

It is important to have the domains for the variables $r, t$, since finding the minimum of $E$ would require some constraints. The ranges for $r$ and $t$ can be determined by analyzing the square root from $t_1$ given by Eq.\eqref{terms}.

In order for the function $E$ to be real, the term in the square root must be non-negative, thus:
$$rt(2I-r)(2S-t)\geq 0\ ,$$
which provides the valid regions for $r, t$ as:
$$r\in [0, 2I]\ , \ t\in [0, 2S]$$

\subsection{Critical points}

We solve via \texttt{NSolve} the system of equations from Eq.\eqref{system-eq} and only consider the real solutions. The points provided by the system are denoted by $\{r_c, t_c\}$ (to emphasize the critical points). Moreover, by studying second-order partial derivatives of $E$ w.r.t to $r$ and $t$, we can finally determine if the critical points represent local minimum or maximum.

\subsection*{Numerical Analysis}

\begin{table}[h]
    \centering
    \begin{tabular}{ccccc}
        \toprule
        $I$ & $r_c$ & $t_c$ & $\frac{\partial^2 E}{\partial r^2}$ & $\frac{\partial^2 E}{\partial t^2}$ \\
        \midrule
        1  & 0.8     & 0.8     & 2.08333  & 2.08333  \\
        2  & 1.80364 & 0.767656 & 0.986868  & 4.32622  \\
        3  & 2.80429 & 0.761959 & 0.651659  & 6.53469  \\
        4  & 3.80452 & 0.759986 & 0.487129  & 8.73419  \\
        5  & 4.80462 & 0.759076 & 0.389108  & 10.9301  \\
        6  & 5.80468 & 0.758583 & 0.323989  & 13.1241  \\
        7  & 6.80471 & 0.758286 & 0.277566  & 15.3171  \\
        8  & 7.80473 & 0.758093 & 0.242792  & 17.5095  \\
        9  & 8.80475 & 0.757961 & 0.215767  & 19.7014  \\
        10 & 9.80476 & 0.757867 & 0.19416  & 21.893  \\
        \bottomrule
    \end{tabular}
    \caption{Critical points and second-order derivatives for fixed $S=1$}
    \label{table1}
\end{table}

\begin{table}[h]
    \centering
    \begin{tabular}{ccccc}
        \toprule
        $S$ & $r_c$ & $t_c$ & $\frac{\partial^2 E}{\partial r^2}$ & $\frac{\partial^2 E}{\partial t^2}$ \\
        \midrule
        1  & 0.8     & 0.8     & 2.08333  & 2.08333  \\
        2  & 0.767656 & 1.80364 & 4.32622  & 0.986868  \\
        3  & 0.761959 & 2.80429 & 6.53469  & 0.651659  \\
        4  & 0.759986 & 3.80452 & 8.73419  & 0.487129  \\
        5  & 0.759076 & 4.80462 & 10.9301  & 0.389108  \\
        6  & 0.758583 & 5.80468 & 13.1241  & 0.323989  \\
        7  & 0.758286 & 6.80471 & 15.3171  & 0.277566  \\
        8  & 0.758093 & 7.80473 & 17.5095  & 0.242792  \\
        9  & 0.757961 & 8.80475 & 19.7014  & 0.215767  \\
        10 & 0.757867 & 9.80476 & 21.893  & 0.19416  \\
        \bottomrule
    \end{tabular}
    \caption{Critical points and second-order derivatives for fixed $I=1$}
    \label{table2}
\end{table}

As we can see from Table \ref{table1} and \ref{table2} the critical points are all \textbf{minima}, since the second derivative is positive. The value for $(r_c, t_c)$ for fixed spin values $I,S=8, 8$ is the following tuple:
$$(r_c, t_c;I=8,S=8)=(7.7509\ ,\ 7.7509)$$
\end{document}