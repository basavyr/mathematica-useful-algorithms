
\documentclass[a4paper]{article}
\usepackage[margin=1cm]{geometry}
\usepackage{amsmath}
\usepackage{amssymb}
\usepackage{mathtools}


\title{Formal analysis of the $E/A_{-}$ function}
\author{}
\date{}

\begin{document}

\maketitle

\section{Function Definition}

We will simply denote $E/A_{-}$ by $E$ throughout this document. The function is defined as:
\begin{align}
E(r,t,I,S) &= t_1(r,t,I,S) + t_2(r,t,I,S)\\
\end{align}
where:
\begin{align}
t_1(r,t,I,S) &= -2\sqrt{rt(2I-r)(2S-t)}\\
t_2(r,t,I,S) &= \frac{I-1}{2S}t + \frac{S-1}{2I}r + \frac{r}{2IS}t
\label{terms}
\end{align}

For fixed parameters $I$ and $S$, we can express the function as:
\begin{align}
E(r,t) &= -2\sqrt{rt(2I-r)(2S-t)} + \frac{I-1}{2S}t + \frac{S-1}{2I}r + \frac{r}{2IS}t
\end{align}

\section{Critical Points}
The critical points of the $E$ function are found by setting the first-order partial derivatives equal to zero:
\begin{align}
\frac{\partial E}{\partial r} &= \frac{-t(2I-r)(2S-t)}{2r\sqrt{rt(2I-r)(2S-t)}} + \frac{S-1}{2I} + \frac{t}{2IS} = 0\ ,\\
\frac{\partial E}{\partial t} &= \frac{-r(2I-r)(2S-t)}{2t\sqrt{rt(2I-r)(2S-t)}} + \frac{I-1}{2S} + \frac{r}{2IS} = 0
\end{align}
These equations determine the critical points $(r_c, t_c)$ where the gradient of the function is zero.

\section{Hessian Matrix}

The Hessian matrix of the $E$ function with respect to variables $r$ and $t$ (treating $I$ and $S$ as fixed parameters) is:

\begin{align}
H_{E}(r,t) = 
\begin{pmatrix}
\frac{\partial^2 E}{\partial r^2} & \frac{\partial^2 E}{\partial r \partial t} \\
\frac{\partial^2 E}{\partial t \partial r} & \frac{\partial^2 E}{\partial t^2}
\end{pmatrix}
= 
\begin{pmatrix}
\frac{t(2S-t)(3r-2I)}{4r(r-2I)^2\sqrt{rt(2I-r)(2S-t)}} & \frac{1}{2IS} - \frac{(2I-r)(2S-t) + rt}{4rt\sqrt{rt(2I-r)(2S-t)}} \\
\frac{1}{2IS} - \frac{(2I-r)(2S-t) + rt}{4rt\sqrt{rt(2I-r)(2S-t)}} & \frac{r(2I-r)(3t-2S)}{4t(t-2S)^2\sqrt{rt(2I-r)(2S-t)}}
\end{pmatrix}
\end{align}

\section{Function Domain}

It is important to have the domains for the variables $r, t$, since finding the minimum of $E$ would require some constraints. The ranges for $r$ and $t$ can be determined by analyzing the square root from $t_1$ given by Eq.\eqref{terms}.

In order for the function $E$ to be real, the term in the square root must be non-negative, thus:
$$rt(2I-r)(2S-t)\geq 0\ ,$$
which provides the valid regions for $r, t$ as:
$$r\in [0, 2I]\ , \ t\in [0, 2S]$$

\subsection{Critical points}

\end{document}