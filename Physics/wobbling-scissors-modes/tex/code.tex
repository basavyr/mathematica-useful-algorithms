
\documentclass[a4paper]{article}
\usepackage[margin=1cm]{geometry}
\usepackage{amsmath}
\usepackage{amssymb}
\usepackage{mathtools}
\usepackage{booktabs}


\title{Formal analysis of the $E/A_{-}$ function}
\author{}
\date{}

\begin{document}

\maketitle

\section{Function Definition}

We will simply denote $E/A_{-}$ by $E$ throughout this document. The function is defined as:
\begin{align}
E(r,t,I,S) &= t_1(r,t,I,S) + t_2(r,t,I,S)
\end{align}
where:
\begin{align}
t_1(r,t,I,S) &= -2\sqrt{rt(2I-r)(2S-t)}\ ,\nonumber \\
t_2(r,t,I,S) &= \frac{I-1}{2S}t + \frac{S-1}{2I}r + \frac{rt}{4IS}\ .
\label{terms}
\end{align}

For fixed parameters $I$ and $S$, we can express the function as:
\begin{align}
E(r,t) &= -2\sqrt{rt(2I-r)(2S-t)} + \frac{I-1}{2S}t + \frac{S-1}{2I}r + \frac{rt}{4IS}
\end{align}

\section{Critical Points}
The critical points of the $E$ function are found by setting the first-order partial derivatives equal to zero:
\begin{align}
\frac{\partial E}{\partial r} &= \frac{S - 1}{2I} + \frac{t}{4IS} - \frac{(2S - t)t(2I - r) - (2S - t)t r}{\sqrt{(2S - t)t(2I - r)r}} = 0\ ,\nonumber\\
\frac{\partial E}{\partial t} &= \frac{I - 1}{2S} + \frac{r}{4IS} - \frac{(2I - r)r(2S - t) - (2I - r)r t}{\sqrt{(2I - r)r(2S - t)t}} = 0
\label{system-eq}
\end{align}
These equations determine the critical points $(r_c, t_c)$ where the gradient of the function is zero.

\section{Hessian Matrix}

The Hessian matrix of the $E$ function with respect to variables $r$ and $t$ (treating $I$ and $S$ as fixed parameters) is:
\begin{align}
H_{E}(r,t) = 
\begin{pmatrix}
\frac{\partial^2 E}{\partial r^2} & \frac{\partial^2 E}{\partial r \partial t} \\
\frac{\partial^2 E}{\partial t \partial r} & \frac{\partial^2 E}{\partial t^2}
\end{pmatrix}\ ,
\end{align}
with the derivatives:
\begin{align}
    \frac{\partial^2 E}{\partial r^2} &= -2 \left( - \frac{(2S - t) t}{\sqrt{(2S - t) t (2I - r) r}} - \frac{((2S - t) t (2I - r) - (2S - t) t r)^2}{4 \left( (2S - t) t (2I - r) r \right)^{3/2}} \right) \\
    \frac{\partial^2 E}{\partial t^2} &= -2 \left( - \frac{(2I - r) r}{\sqrt{(2I - r) r (2S - t) t}} - \frac{((2I - r) r (2S - t) - (2I - r) r t)^2}{4 \left( (2I - r) r (2S - t) t \right)^{3/2}} \right) \\
    \frac{\partial^2 E}{\partial r \partial t} &= \frac{1}{4IS} - \frac{(2I - r)(2S - t) - r(2S - t) - (2I - r) t + rt}{\sqrt{(2I - r) r (2S - t) t}}
\end{align}

\section{Function Domain}

It is important to have the domains for the variables $r, t$, since finding the minimum of $E$ would require some constraints. The ranges for $r$ and $t$ can be determined by analyzing the square root from $t_1$ given by Eq.\eqref{terms}.

In order for the function $E$ to be real, the term in the square root must be non-negative, thus:
$$rt(2I-r)(2S-t)\geq 0\ ,$$
which provides the valid regions for $r, t$ as:
$$r\in [0, 2I]\ , \ t\in [0, 2S]$$

\subsection{Critical points}

We solve via \texttt{NSolve} the system of equations from Eq.\eqref{system-eq} and only consider the real solutions. The points provided by the system are denoted by $\{r_c, t_c\}$ (to emphasize the critical points). Moreover, by studying second-order partial derivatives of $E$ w.r.t to $r$ and $t$, we can finally determine if the critical points represent local minimum or maximum.

\section{Numerical Analysis}

All the numerical values for the minimum points are provided below.

\begin{table}[h]
    \centering
    \begin{tabular}{ccccc}
        \toprule
        $I$ & $r_c$ & $t_c$ & $\frac{\partial^2 E}{\partial r^2}$ & $\frac{\partial^2 E}{\partial t^2}$ \\
        \midrule
        1  & 0.888889 & 0.888889 & 2.025 & 2.025 \\
        2  & 1.89609  & 0.81856  & 0.987397 & 4.20029 \\
        3  & 2.89829  & 0.79726  & 0.653949 & 6.38622 \\
        4  & 3.89935  & 0.786972 & 0.488987 & 8.57447 \\
        5  & 4.89998  & 0.780911 & 0.390516 & 10.7636 \\
        6  & 5.90039  & 0.776915 & 0.325067 & 12.9532 \\
        7  & 6.90068  & 0.774083 & 0.278412 & 15.143 \\
        8  & 7.9009   & 0.771971 & 0.24347  & 17.3329 \\
        9  & 8.90107  & 0.770335 & 0.216321 & 19.523 \\
        10 & 9.9012   & 0.769031 & 0.194621 & 21.7131 \\
        \bottomrule
    \end{tabular}
    \caption{Critical points and second-order derivatives for fixed $S=1$.}
    \label{table1}
\end{table}

\begin{table}[h]
    \centering
    \begin{tabular}{ccccc}
        \toprule
        $S$ & $r_c$ & $t_c$ & $\frac{\partial^2 E}{\partial r^2}$ & $\frac{\partial^2 E}{\partial t^2}$ \\
        \midrule
        1  & 0.888889 & 0.888889 & 2.025 & 2.025 \\
        2  & 0.81856  & 1.89609  & 4.20029 & 0.987397 \\
        3  & 0.79726  & 2.89829  & 6.38622 & 0.653949 \\
        4  & 0.786972 & 3.89935  & 8.57447 & 0.488987 \\
        5  & 0.780911 & 4.89998  & 10.7636 & 0.390516 \\
        6  & 0.776915 & 5.90039  & 12.9532 & 0.325067 \\
        7  & 0.774083 & 6.90068  & 15.143  & 0.278412 \\
        8  & 0.771971 & 7.9009   & 17.3329 & 0.24347 \\
        9  & 0.770335 & 8.90107  & 19.523  & 0.216321 \\
        10 & 0.769031 & 9.9012   & 21.7131 & 0.194621 \\
        \bottomrule
    \end{tabular}
    \caption{Critical points and second-order derivatives for fixed $I=1$.}
    \label{table2}
\end{table}

As we can see from Table \ref{table1} and \ref{table2} the critical points are all \textbf{minima}, since the second derivative is positive. The value for $(r_c, t_c)$ for fixed spin values $I,S=8, 8$ is the following tuple:
$$(r_c, t_c;I=8,S=8)=(7.76608\ ,\ 7.76608)$$
\end{document}